
\keywords{GraphQL, REST, Representational State Transfer, Web APIs, API, Ohjelmointirajapinnat, Ohjelmistoarkkitehtuurit}

\keywordstwo{GraphQL, REST, Representational State Transfer, Web APIs, APIs, Application programming interfaces, Software architecture}

\setlength{\parindent}{0em}
\setlength{\parskip}{1em}

\begin{abstract}

Jokaisen hieman monimutkaisemman verkkosivun takana on lähes poikkeuksetta sitä palveleva rajapinta, josta sivu voi hakea informaatiota näytettäväksi käyttäjilleen. Tehokkaiden rajapintojen rakentaminen ei ole kuitenkaan helppoa ja kehittäjien on aina ollut vaikea valita erilaisten arkkitehtuurimallien välillä. Representational State Transfer (REST) -arkkitehtuuri web-rajapintojen toteutukseen on ollut jo pitkään monen rajapinnanrakentajan suosikki. Vuonna 2012 Facebook kehitti GraphQL:n ja uuden tavan hakea dataa web-rajapinnoista sen tarjoamalla kyselykielellä. Sittemmin GraphQL on hiljalleen saanut enemmän ja enemmän jalansijaa web-rajapintojen toteutuksissa. Jopa REST-rajapintojen kuolemaa on ennustettu, mutta muutos ei näytä tapahtuvan niin nopeasti kuin on oletettu. 

Tämä tutkielma vertailee kahta tällä hetkellä suosittua tapaa luoda asiakas- palvelinsovellusten rajapintoja: GraphQL ja REST. Tämä tutkielma pyrkii etenkin selvittämään kirjallisuuskatsauksen avulla, voiko GraphQL ottaa REST:n paikan suosituimpana rajapintojen toteutustapana lähivuosina. REST:iä ja GraphQL:ää vertailevaa tutkimustyötä ei olla toistaiseksi tehty kovinkaan paljoa. Olemassa oleva tutkimustyö on keskittynyt rajapintojen kehitystyön helppouteen, turvallisuuten ja kehittäjien mielipiteisiin. Tulosten vahvistamiseksi olisi hyvä tehdä lisää tutkimustyötä tulevaisuudessa.

GraphQL:n saavuttamasta suhteellisen nopeasta suosiosta huolimatta, se ei vaikuta pystyvän syrjäyttämään REST-ekosysteemiä vielä lähivuosina. GraphQL:n tarjoama kyselykieli antaa joustavuutta, mutta tekee rajapinnoista myös hieman monimutkaisempia. On otettava myös huomioon, että REST on muodostunut muotisanaksi, millä viitataan myös REST:n kaltaisiin rajapintoihin. Näin ollen GraphQL saattaa olla jo suositumpi kuin todelliset, alkuperäisen määritelmän mukaiset REST-toteutukset.

\end{abstract}

\begin{abstracten}
Behind every slightly more complex web page, there is almost invariably a application programming interface (API) from which the page can retrieve information to display to its users. However, building efficient APIs is not easy and developers have always had to choose between different architectural models. The Representational State Transfer (REST) architecture for implementing web APIs has long been a favourite of many API builders. In 2012, Facebook developed GraphQL and a new way to retrieve data from APIs of the web using the query language it provides. Since then, GraphQL has slowly gained more and more ground in web API implementations. Even the death of REST APIs has been predicted, but the change does not seem to be happening as fast as expected. 

This thesis compares two currently popular ways to create client-server APIs: GraphQL and REST. In particular, by means of a literature review, this thesis aims to determine whether GraphQL can take the place of REST as the most popular API implementation method in the coming years. Not much research has been done yet comparing REST and GraphQL. Existing research has focused on the ease of API development, security and developer perceptions. More research should be done in the future to confirm the results.

Despite GraphQL's relatively rapid rise in popularity, it does not appear that it will be able to displace the REST ecosystem for years to come. The query language provided by GraphQL gives flexibility, but also makes the APIs a bit more complex. It should also be noted that REST has become a buzzword, which also refers to REST-like APIs. Thus GraphQL may already be more popular than actual REST implementations as originally defined by Roy Fielding.
\end{abstracten}