\documentclass[a4paper,12pt,language=finnish,version=final,hidechapters=true,sharelatex=true,emptyfirstpages=true,minted=true]{utuftthesis}

% Hyperref ladataan lopputyöpohjassa, ei kannata ladata täällä - paketti on hieman nirso siitä, milloin ja kuinka monta kertaa se ladataan
\usepackage{hyperref}

% Älä käytä pdftitleä täällä - aseta metadata output.xmpdata-tiedostossa. Kun version laitetaan finaliksi, sen metadata pdf:ään
\hypersetup{
    colorlinks=true,
    linkcolor=black,
    filecolor=magenta,      
    urlcolor=blue,
    %pdftitle={Overleaf Example},
    %pdfpagemode=FullScreen,
    }

%% Hieman paremmat oletukset mintedille
\setminted{
breaklines,
baselinestretch=1.1,
frame=lines,
linenos
}
    
\urlstyle{same}

\setcounter{secnumdepth}{2}
\setcounter{tocdepth}{2}

\addbibresource{Bibliografia.bib}
\begin{document}

\pubyear{2022}

\pubmonth{5}

\publab{Ohjelmistotekniikka}

\publaben{Software Engineering}

\pubtype{tkk}
\title{GraphQL ja REST: tuleeko GraphQL ottamaan REST:n paikan web-rajapintojen kuninkaana?}
\author{Juhana Kuparinen}

\maketitle

\keywords{GraphQL, REST, Representational State Transfer, Web APIs, API, Ohjelmointirajapinnat, Ohjelmistoarkkitehtuurit}

\keywordstwo{GraphQL, REST, Representational State Transfer, Web APIs, APIs, Application programming interfaces, Software architecture}

\setlength{\parindent}{0em}
\setlength{\parskip}{1em}

\begin{abstract}

Jokaisen hieman monimutkaisemman verkkosivun takana on lähes poikkeuksetta sitä palveleva rajapinta, josta sivu voi hakea informaatiota näytettäväksi käyttäjilleen. Tehokkaiden rajapintojen rakentaminen ei ole kuitenkaan helppoa ja kehittäjien on aina ollut vaikea valita erilaisten arkkitehtuurimallien välillä. Representational State Transfer (REST) -arkkitehtuuri web-rajapintojen toteutukseen on ollut jo pitkään monen rajapinnanrakentajan suosikki. Vuonna 2012 Facebook kehitti GraphQL:n ja uuden tavan hakea dataa web-rajapinnoista sen tarjoamalla kyselykielellä. Sittemmin GraphQL on hiljalleen saanut enemmän ja enemmän jalansijaa web-rajapintojen toteutuksissa. Jopa REST-rajapintojen kuolemaa on ennustettu, mutta muutos ei näytä tapahtuvan niin nopeasti kuin on oletettu. 

Tämä tutkielma vertailee kahta tällä hetkellä suosittua tapaa luoda asiakas- palvelinsovellusten rajapintoja: GraphQL ja REST. Tämä tutkielma pyrkii etenkin selvittämään kirjallisuuskatsauksen avulla, voiko GraphQL ottaa REST:n paikan suosituimpana rajapintojen toteutustapana lähivuosina. REST:iä ja GraphQL:ää vertailevaa tutkimustyötä ei olla toistaiseksi tehty kovinkaan paljoa. Olemassa oleva tutkimustyö on keskittynyt rajapintojen kehitystyön helppouteen, turvallisuuten ja kehittäjien mielipiteisiin. Tulosten vahvistamiseksi olisi hyvä tehdä lisää tutkimustyötä tulevaisuudessa.

GraphQL:n saavuttamasta suhteellisen nopeasta suosiosta huolimatta, se ei vaikuta pystyvän syrjäyttämään REST-ekosysteemiä vielä lähivuosina. GraphQL:n tarjoama kyselykieli antaa joustavuutta, mutta tekee rajapinnoista myös hieman monimutkaisempia. On otettava myös huomioon, että REST on muodostunut muotisanaksi, millä viitataan myös REST:n kaltaisiin rajapintoihin. Näin ollen GraphQL saattaa olla jo suositumpi kuin todelliset, alkuperäisen määritelmän mukaiset REST-toteutukset.

\end{abstract}

\begin{abstracten}
Behind every slightly more complex web page, there is almost invariably a application programming interface (API) from which the page can retrieve information to display to its users. However, building efficient APIs is not easy and developers have always had to choose between different architectural models. The Representational State Transfer (REST) architecture for implementing web APIs has long been a favourite of many API builders. In 2012, Facebook developed GraphQL and a new way to retrieve data from APIs of the web using the query language it provides. Since then, GraphQL has slowly gained more and more ground in web API implementations. Even the death of REST APIs has been predicted, but the change does not seem to be happening as fast as expected. 

This thesis compares two currently popular ways to create client-server APIs: GraphQL and REST. In particular, by means of a literature review, this thesis aims to determine whether GraphQL can take the place of REST as the most popular API implementation method in the coming years. Not much research has been done yet comparing REST and GraphQL. Existing research has focused on the ease of API development, security and developer perceptions. More research should be done in the future to confirm the results.

Despite GraphQL's relatively rapid rise in popularity, it does not appear that it will be able to displace the REST ecosystem for years to come. The query language provided by GraphQL gives flexibility, but also makes the APIs a bit more complex. It should also be noted that REST has become a buzzword, which also refers to REST-like APIs. Thus GraphQL may already be more popular than actual REST implementations as originally defined by Roy Fielding.
\end{abstracten}

% empty pagestyle for table of contents etc.
% otherwise you'll get simple page style with roman page numbers
\pagestyle{empty}

% mandatory
\tableofcontents

% if you want a list of figures
%\listoffigures

% if you want a list of tables
%\listoftables

% 'list of acronyms'
%   - you may not need this at all
%   - create a chapter called List Of Acronyms (or whatever), which
%     should contain all your acronym definitions, e.g. 
%     \chapter{List Of Acronyms} 
%   - the secnumdepth trickery is needed because acronyms are as a
%     standard chapter and we are faking '\listofacronyms'
%
%\setcounter{secnumdepth}{-1}
%\input{your acronym chapter's file name}
%\setcounter{secnumdepth}{2}% setup page numbering, page counter, etc.%
\begin{comment}
The thesis starts here.

To better organize things, create a new tex file for each chapter
and input it below.

Avoid using the å, ä, ö or <space> characters in referred names and
underscores \_ in file names (may break hyperref).

Good luck!
\end{comment}

\chapter{Johdanto} \label{Johdanto}

\setlength{\parindent}{0em}
\setlength{\parskip}{1em}

Roy Fielding esitti vuonna 2000 väitöskirjassaan \textit{Architectural Styles and the Design of Network-based Software Architectures} \textbf{REST} (Representational State Transfer) rajapinta-arkkitehtuurin. REST määritteli, miten dataa (REST-maailmassa resursseja), kuten HTML, JSON ja XML, tulisi jakaa WWW:n kaltaisessa suuressa hypermediasysteemissä \cite{FieldingRThesis}. Läpi 90-luvun oli suuri tarve määritellä yhteinen standardi verkon rajapinnoille, jonka ratkaisuksi Fielding yritti tarjota REST:iä \cite{FactsAboutW3C}. Sittemmin REST on noussut hyvin suosituksi tavaksi luoda rajapintoja web-ympäristössä mm. helpon toteuttettavuudensa takia.

Facebookin (nykyisin Meta) sisäisesti vuonna 2012 kehittämä GraphQL on noussut viime vuosina REST:n valta-aseman todennäköisimmäksi haastajaksi \cite{graphql-intro}. Jopa REST:n \href{https://www.stridenyc.com/podcasts/52-is-2018-the-year-graphql-kills-rest}{kuolemaa} on ennustettu aika ajoin, mutta GraphQL:n voitto kamppailussa ei näytä vielä olevan aivan täysin taattua \cite{the-year-graphql-kills-rest}. Kysymys kuuluukin: johtaako GraphQL todellakin REST-arkkitehtuurin käytön hiipumiseen vai tulevatko nämä elämään rinnakkain vaihtoehtoisina tapoina toteuttaa rajapintoja verkossa?

%--------------- KAPPALE ---------------------%

\section{Tutkimuksen rajaus}

Tässä kandidaatintutkielmassa tarkastellaan kahta web rajapinta-arkkitehtuurimallia pääasiallisesti HTTP-protokollan yli toteutettavalle kommunikoinnille: \textbf{GraphQL} ja \textbf{REST}. Muitakin samanlaisia arkkitehtuurimalleja kuten \textbf{SOAP} tai \textbf{RPC} on olemassa, mutta näitä malleja ei käydä läpi tässä tutkielmassa. Tämä työ keskittyy puhtaasti REST:n ja GraphQL:n vertailuun, sillä ne ovat tällä hetkellä ajankohtaisimmat ja suosituimmat arkkitehtuurimallit, verkon asiakas-palvelinsovelluksia rakennettaessa.

GraphQL ja REST ovat jo itsessään niin laajoja käsitteitä, että niistä voisi tehdä kokonaan omat tutkielmansa. Tässä työssä pyritään käymään läpi REST ja GraphQL vain vertailulle oleellisella tavalla. Esimerkiksi GraphQL-moottorin tarkempi toteutus on jätetty tästä tutkielmasta kokonaan pois. REST:iin ja GraphQL:ään voi syvällisemmin tutustua esimerkiksi Roy Fieldingin \href{https://www.ics.uci.edu/~fielding/pubs/dissertation/top.htm}{väitöskirjassa}\footnote{https://www.ics.uci.edu/~fielding/pubs/dissertation/top.htm} (REST) ja GraphQL:n virallisilla \href{https://graphql.org/}{sivuilla}\footnote{https://graphql.org/} sekä GraphQL:n \href{https://spec.graphql.org/}{spesifikaatioissa}\footnote{https://spec.graphql.org/} (GraphQL).

%--------------- KAPPALE ---------------------%

\section{Tutkimuskysymykset}\label{Alaotsikko}

Työn rajauksen perusteella asetetttiin seuraavat tutkimuskysymykset:

\subsubsection{Mitä hyötyjä ja haittoja on GraphQL:n käytössä verrattuna REST:iin?}

Tämän tutkimuskysymyksen puitteissa pyritään vertailemaan GraphQL:ää ja REST:iä. Vertailu luo pohjan työn lopputulokselle ja seuraavaksi esiteltävän, työn kannalta oleellisimman, tutkimuskysymyksen vastaamiseen.


\subsubsection{Tuleeko GraphQL korvaamaan REST-rajapinnat?}

Tämän kandityön lopputuloksen kannalta tärkein kysymys. Tässä työssä halutaan selvittää, tuleeko GraphQL ylittämään REST:n suosion ratkaisuna verkon rajapinnoille.


%--------------- KAPPALE ---------------------%

\section{Tutkimusmenetelmät ja lähteet}

Kun käsitellään verkkoympäristössä eläviä, nopeasti kehittyviä teknologioita ja arkkitehtuurimalleja, uusimman ja luotettavimman tiedon löytää usein aihetta edustavilta virallisilta verkkosivuilta. Itse REST ja GraphQL-rajapintoja vertailevaa tutkimustyötä ole tuotettu vielä kovinkaan paljoa ja näin ollen työssä käytettävien tieteellisten lähteiden määrä on vähäisempi ja niiden esittämiin väitteisiin on suhtauduttava suuremmalla varauksella kuin vastaavissa tutkielmissa yleensä.

Tämä työ käyttää REST:iä esittävän informaation pääasiallisena lähteenä Roy Fielding väitöskirjaa \textit{Architectural Styles and the Design of Network-based Software Architectures} \cite{FieldingRThesis}. GraphQL:ää edustava tieto on pääasiassa peräisin graphql.org-verkkosivulta \cite{graphqlorg}. Lisäksi näiden tukena käytetään lähteenä web-standardien saralla luotettaviksi todettuja verkkolähteitä kuten MDN Webdocs ja w3.org \cite{mdn, FactsAboutW3C}. Tieteellisiä lähteitä on haettu Google Scholarista ja IEEE:n sekä ACM:n tietokannoista.\chapter{GraphQL ja REST} \label{toinenluku}

Ennen GraphQL- ja REST-rajapintojen varsinaista vertailua, tässä luvussa käydään läpi molemmat rajapinta-arkkitehtuurit pintapuolisesti. Näin myös lukijat, jotka eivät tiedä paljoakaan aiheesta, saavat pienen perehdytyksen käsiteltäviin asioihin. GraphQL:n ja REST:in lisäksi aloitetaan luku avaamalla rajapinta-sanan merkitys käsiteltävien aiheiden yhteydessä.



\setlength{\parindent}{0em}
\setlength{\parskip}{1em}

\section{Ohjelmointirajapinnat}
\label{Ohjelmointirajapinnat}

Ohjelmointirajapinta (engl. API, Application Programming Interface) on joukko tietokoneohjelman ominaisuuksia ja sääntöjä, joiden kautta muut ohjelmistot voivat olla vuorovaikutuksessa tietokoneohjelman kanssa \cite{mdnapi}. Usein rajapinta määrittelee tiettyjä alkuehtoja, joiden täyttyessä se suorittaa tietyn tehtävän ja palauttaa määrämuotoisen tuloksen. Tiivistettynä ohjelmiston rajapinnan kautta voidaan kommunikoida kyseisen ohjelman kanssa. Datan hakeminen ja tilan muuttaminen kohdeohjelmistossa ovat yleisiä käyttötilanteita.

Web-ympäristössä rajapinnat ovat usein joukko koodin ominaisuuksia, kuten metodeja, tapahtumia ja URL-osoitteita \cite{mdnapi}. Web-kehittäjät käyttävät näitä sovelluksissaan kommunikoidakseen selainten sekä kolmansien osapuolten verkko-osoitteiden ja sivujen kanssa.

\section{GraphQL}
\label{GraphQL}

GraphQL on asiakasohjelmien rakentamiseen kehitetty avoimen lähdekoodin kyselykieli ja ajonaikainen ympäristö palvelinsovelluksille. Se tarjoaa intuitiivisen ja joustavan syntaksin sekä systeemin kuvaamaan ohjelman data- ja vuorovaikutusvaatimukset. Kyselykielenä GraphQL ei ole perinteinen Turing-täydellinen ohjelmointikieli, joka
kykenisi mielivaltaiseen laskentaan: se on luotu kuvaamaan pyyntöjä erilaisiin ohjelmointirajapintoihin. GraphQL ei esitä vaatimuksia, kuten tiettyä kieltä tai tietokantajärjestelmää, sen toteuttavalle ohjelmalle. GraphQL:n toteuttavat sovelluspalvelut liittävät itsensä sen tarjoamaan yhtenäiseen kieleen ja tyyppijärjestelmään. \cite{graphql-spec, graphqlorg}

\subsection{Skeemat ja tyypit}
\label{Skeemat ja tyypit}

GraphQL-filosofian ytimessä on tarjota dataa hakevalle asiakkaalle mahdollisuus määritellä tarkasti, mitä dataa rajapinta palauttaa. GraphQL kuvaa rajapinnan datan oman tyyppisysteeminsä ja skeemojen avulla graafin muotoisena tietorakenteena. GraphQL määrittelee nämä tietorakenteet oman tyyppikielensä avulla \textit{(GraphQL skeemakieli - GraphQL schema language)}. \cite{SchemasAndTypes} 

GraphQL nimeää kuusi erilaista tyyppimääritelmää (type definitions). Näiden lisäksi on olemassa kaksi niin sanottua "kuorityyppiä" (wrapper types) \cite{graphql-spec}. Käydään seuraavaksi näistä jokainen lyhyesti läpi. 

\paragraph{Oliotyypit} GraphQL-skeemojen yksinkertaisin komponentti on oliotyyppi (object type). Oliotyypillä voidaan kuvata, missä muodossa data tulee palautumaan rajapinnasta. GraphQL-skeemakielellä voidaan määritellä oliotyyppi esimerkiksi seuraavasti:

\inputminted{gql.py:GraphQLLexer -x}{listaukset/character.graphql}

Jokaiselle oliotyypille määritellään kentät. Kenttä voi olla joko yksi GraphQL:n sisäänrakennetuista (tai itse luoduista) skalaarityypeistä tai toinen oliotyyppi. Jokainen kenttä voi myös ottaa vastaan argumentteja (esimerkin kenttä \texttt{height}). Argumentit voivat olla joko valinnaisia tai pakollisia. Valinnaisille argumenteille voidaan määritellä oletusarvo, joka on asetettu esimerkin kentän \texttt{height} argumentille \texttt{unit} olemaan \texttt{METER}. \cite{SchemasAndTypes}

\paragraph{Skalaarityypit} GraphQL-kyselyn palauttaman graafin on esitettävä kentät oikeanlaisena datana graafin päätepisteissä tai "lehdissä". GraphQL kuvaa datan, kuten merkkijonot tai numerot, skalaarityyppien avulla. GraphQL:ssä on sisään määriteltynä viisi skalaarityyppiä: \textbf{Int} eli kokonaisluvut (signed 32-bit), \textbf{Float} eli liukuluvut, \textbf{String} eli merkkijonot, \textbf{Boolean} eli totuusarvot true tai false ja \textbf{ID}. ID kuvaa uniikkia tunnistetta. Se serialisoidaan samoin kuin String, mutta ei ole tarkoitettu ihmisten luettavaksi. \cite{SchemasAndTypes}

Useimmissa GraphQL-palvelutoteutuksissa voidaan myös määritellä omatekoisia skalaarityyppejä. Omien skalaarityyppien serialisoinnin, deserialisoinnin ja validoinnin toteuttaminen jätetään ohjelmoijalle itselleen. \cite{SchemasAndTypes}

\paragraph{Rajapinta} Rajapinta GraphQL-skeemakielen yhteydessä muistuttaa olio-ohjelmoinnista tuttua käsitettä. Rajapinta on niin sanottu abstrakti tyyppi ja toinen abstrakteista tyypeistä, joka on määritelty GraphQL:ssä. Rajapinta sisältää tietyn joukon kenttiä, joiden täytyy löytyä sen toteuttavilta oliotyypeiltä. \cite{SchemasAndTypes, graphql-spec} Rajapintoja voitaisiin määritellä ja käyttää esimerkiksi seuraavasti:

\inputminted{gql.py:GraphQLLexer -x}{listaukset/interface.graphql}

\paragraph{Unionit} GraphQL-unioni edustaa oliota, joka voi olla mikä tahansa oliotyyppi listassa tyyppejä ja on GraphQL:n toinen abstrakti tyyppi. Unioni ei voi sisältää skalaarityyppejä, kuten String tai Int: sitä käytetään tiukasti oliotyyppien kanssa. Unionit muistuttavat rajapintoja, mutta eivät määrittele yhteisiä kenttiä tyyppien välillä. \cite{graphql-spec}

\inputminted{gql.py:GraphQLLexer -x}{listaukset/union.graphql}

Esimerkin tyypin \textit{SearchQuery}-kentän \textit{firstSearchResult}-arvo voisi olla joko Photo tai Person.

\paragraph{Luetellut tyypit} Luetellut tyypit tai enumit (engl. enumeration type) edustavat erityistä skalaarityyppiä, jonka arvo voi olla mikä tahansa tietyn joukon arvoista. Näiltä osin enumit eivät eroa GraphQL:ssä muiden ohjelmointikielien vastaavista rakenteista. Enum-rakenteen voi määritellä seuraavasti: \cite{graphql-spec, SchemasAndTypes}

\inputminted{gql.py:GraphQLLexer -x}{listaukset/enum.graphql}

\paragraph{Syöteoliotyypit} Joissakin tapauksissa halutaan syöttää kenttien argumenteiksi hieman monimutkaisempia tietorakenteita. Tällöin on käytettävä syöteoliotyyppiä (input object type). Syöteolio voidaan määritellä seuraavasti:

\inputminted{gql.py:GraphQLLexer -x}{listaukset/input.graphql}

Syöteolio muistuttaa rakenteeltaan hyvin paljon normaalia oliotyyppiä. Syöteolioissa ei kuitenkaan voi olla kenttiä, jotka sisältävät argumentteja tai viittauksia rajapintoihin tai unioneihin. Normaalia oliotyyppiä ei siis tule antaa argumentin tyypiksi! \cite{graphql-spec}

\paragraph{Kuorityypit: listat ja non null} Edellisissä esimerkeissä joidenkin tyyppien perässä on huutomerkki tai ne ovat hakasulkeiden sisällä. Huutomerkit edustavat ei-nollattavia (non null) kenttiä ja hakasulkeet listoja ja ne määritellään muodossa \textbf{Non-Null} ja \textbf{List} vastaavasti.

GraphQL:ssä kentät voivat saada oletusarvoisesti arvon null. Ei-nollattavan kentän arvo ei voi milloinkaan olla null. Listaksi merkitty kenttä palauttaa taulukon kentän tyyppisiä arvoja.

\inputminted{gql.py:GraphQLLexer -x}{listaukset/wrapper.graphql}

Non-Null- ja List-tyyppisiä kenttiä voidaan yhdistellä, niin kuin edeltävästä koodiesimerkistä nähdään: Hakasulkujen ulkopuolella oleva huutomerkki tarkoittaa, että itse lista ei voi olla null. Hakasulkujen sisällä olevan tyypin perässä oleva huutomerkki taas merkitsee, että listan alkiot eivät voi olla null. Jos molempien perässä on huutomerkit, sekä alkiot että itse lista eivät voi saada arvoa null \cite{graphql-spec, SchemasAndTypes}.



%--------------- KAPPALE ---------------------

\subsection{Kyselyt ja mutaatiot}
\label{Kyselyt ja mutaatiot}

Valmiin GraphQL-rajapinnan tarkoitus on tarjota sitä käyttäville sovelluksille mahdollisuus muokata ja hakea dataa joustavasti ja juuri siten kuin asiakas haluaa. Jos asiakas haluaisi esimerkiksi tietää jonkun tietyn ihmisen nimen ja pituuden, mutta ei mitään muuta tietoa hänestä, haettavasta dataoliosta voidaan valita kentät, jotka rajapinnan tulisi palauttaa kyselyn täytyttyä:

\inputminted{gql.py:GraphQLLexer -x}{listaukset/esimerkki1.graphql}

Vastauksena saataisiin data seuraavassa muodossa:

\inputminted{json}{listaukset/esimerkki1.json}

Vastauksesta huomataan, että palautettu JSON-olio vastaa lähes täysin GraphQL-kyselyn rakennetta. Niin kuin jo aikaisemminkin mainittiin, tämä on GraphQL-filosofian ydin. Ohjelmoija voi aina odottaa saavansa halutun datan rajapinnasta, eikä mitään ylimääräistä. \cite{QueriesAndMutations}

\paragraph{Operaatiotyypit} GraphQL jakaa rajapintoihin tehdyt operaatiot kolmeen luokkaan: kyselyt (query), mutaatiot (mutations) ja tilaukset (subscriptions). Kyselyillä haetaan rajapinnan kautta palvelimella olemassa olevaa dataa. Mutaatioilla voidaan muokata palvelimen tilaa ja siellä sijaitsevaa dataa. Tilauksilla muodostetaan pitkäkestoinen pyyntö palvelimelle, joka hakee dataa vastauksena palvelimen tapahtumiin. Tämä on hyödyllistä esimerkiksi tilanteessa, jossa olisi tarkoitus toteuttaa reaaliajassa toimiva chat-sovellus. \cite{graphql-spec}

Operaatiotyyppi määritellään yleensä GraphQL-operaation yhteydessä ja se on pakollinen, ellei käytetä \textit{kyselyn lyhennettyä syntaksia} (query shorthand syntax). Edellä ollut esimerkki luetaan kyselyksi ja se käyttää lyhennettyä syntaksia. Alla on esimerkki normaalista syntaksista:

\inputminted{gql.py:GraphQLLexer -x}{listaukset/esimerkki3.graphql}

Kyselylle on myös annettu \textbf{operaationimi} \textit{HeroNameAndFriends}. Operaationimi ei ole pakollinen, mutta sen käyttö on suositeltavaa, jotta GraphQL-koodi olisi helpommin luettavaa. Kyselyn lyhennettyä syntaksia ei voi käyttää operaationimien kanssa. \cite{graphql-spec} \cite{QueriesAndMutations}

\paragraph{Argumentit} Edellisessä koodiesimerkissä huomattiin, että pelkän kenttien valitsemisen lisäksi näille voidaan antaa myös argumentteja. Argumentit ja miten niitä määritellään palvelimen skeemoihin, käytiin jo läpi aliluvussa \ref{Skeemat ja tyypit}. Esimerkissä nähtiin, miten näitä käytetään itse kyselyissä. Rajapinnasta haettiin ihmiset, joilla on id 1000 ja haluttiin, että heidän pituutensa näytetään jalkoina. Näin GraphQL:n suoritusympäristö osaa rajata haun koskemaan vain määrittelyä vastaavia tuloksia.

GraphQL:ssä jokainen kenttä ja olio voi saada oman joukon argumentteja. Näin voidaan välttyä usean pyynnön lähettämiseltä rajapinnalle. Tämä on tärkeää REST- ja GraphQL-rajapintojen vertailussa, ja asian merkitystä tullaan käymään tarkemmin läpi, kun näitä arkkitehtuurimalleja vertaillaan. \cite{QueriesAndMutations}

\paragraph{Fragmentit} Kun GraphQL-sovellus kasvaa, ja sitä palvelevien operaatioiden määrä lisääntyy, voi useammassa kyselyssä esiintyä täysin samoja kenttiä. Ohjelmoijasta olisi varmasti turhauttavaa määritellä samat kentät eri kyselyille aina uudestaan. Apuun tulevat fragmentit. Halutut kentät ja oliot voidaan tallettaa fragmentteihin ja näitä voidaan käyttää eri kyselyissä. \cite{QueriesAndMutations} Tätä kuvaa seuraava esimerkki:

\inputminted{gql.py:GraphQLLexer -x}{listaukset/esimerkki2.graphql}

Edeltävään esimerkkiin saadaan seuraava JSON-muotoinen vastaus:

\inputminted{json}{listaukset/esimerkki2.json}

\paragraph{Muuttujat} Luultavasti jokainen GraphQL-operaatioissaan argumentteja käyttävä asiakasohjelma haluaa antaa argumentit dynaamisesti. Dynaamisten arvojen antamiseen on GraphQL:ssä sisäänrakennettu ratkaisu: muuttujat. \cite{QueriesAndMutations} Jos haluataan hakea profiili dynaamisesti määritellyn id:n ja profiilikuvan koon kera jossakin satunnaisessa sovelluksessa, voitaisiin se tehdä seuraavasti:

\inputminted{gql.py:GraphQLLexer -x}{listaukset/esimerkki4.graphql}

Annetut muuttujat JSON-muodossa:

\inputminted{json}{listaukset/esimerkki4-vars.json}

\paragraph{Direktiivit} Muuttujien avulla voidaan antaa argumentteja dynaamisesti kyselyille. GraphQL-rajapintaa käyttävä kehittäjä saattaa myös kohdata tilanteen, jossa itse kyselyn muotoa halutaan muokata dynaamisesti. Tämä voidaan toteuttaa \textbf{direktiiveillä}. Direktiivejä voidaan liittää kenttiin ja fragmentteihin, ja ne vaikuttavat kyselyn suorittamiseen palvelimen haluamalla tavalla. GraphQL-spesifikaatio sisältää kaksi sisäänrakennettua direktiiviä:

\begin{itemize}
	\item \texttt{@include(if: Boolean)} Sisällytä kenttä vastauksessa, mikäli argumentti on tosi.
    \item \texttt{@skip(if: Boolean)} Jätä kenttä pois vastauksesta, mikäli argumentti on tosi.
\end{itemize}

Direktiiviä \texttt{@include} voidaan käyttää esimerkiksi seuraavasti:

\inputminted{gql.py:GraphQLLexer -x}{listaukset/directive.graphql}

Kyselylle annetut muuttujat:

\inputminted{json}{listaukset/directive-vars.json}

Vastauksena saatu JSON-olio:

\inputminted{json}{listaukset/directive-res.json}

Edellä listattujen vakiona GraphQL-toteutuksissa tulevien direktiivien lisäksi eri GraphQL-palvelintoteutuksilla saattaa olla muita omia hyödyllisiä direktiivejä. \cite{QueriesAndMutations}

\paragraph{Mutaatiot} Ohjelmointirajapinnan omaavat tietolähteet tarvitsevat lähes aina datan kyselyn lisäksi tavan muokata dataa. Teknisesti mikä tahansa kysely voisi myös muokata dataa palvelimella. On kuitenkin viisasta noudattaa GraphQL:lle määriteltyä käytäntöä, jonka mukaan dataa palvelimella tulisi muokata ainoastaan mutaatioiden kautta. \cite{QueriesAndMutations}

Rakenteeltaan mutaatiot eivät juurikaan eroa kyselyistä. Jos ne palauttavat jonkin olion, voidaan mutaatioiden rakenne muodostaa täysin samoin kuin kyselynkin, lukuun ottamatta operaationimeä \texttt{mutation}. Tarkastellaan seuraavaksi yksinkertaista mutaatiota:

\inputminted{gql.py:GraphQLLexer -x}{listaukset/mutation.graphql}

Mutaatiolle annetut muuttujat:

\inputminted{json}{listaukset/mutation-vars.json}

Vastauksena saatu JSON-olio:

\inputminted{json}{listaukset/mutation-res.json}

Esimerkistä huomaa, että mutaatio \textit{CreateReviewForAnime} olisi voitu toteuttaa kyselynä tismalleen samalla tavalla. Onko siis mutaatioiden käyttö datan muuttamiseen palvelimella ainoastaan semanttinen käytäntö, jolla ne voidaan erotella helpommin kyselyistä? Jos tarkastellaan mutaatioiden toimintaa konepellin alla, niin vastaus on ei. Verratessa kyselyihin, mutaatioiden kentät suoritetaan peräkkäin, kun taas kyselyiden kentät saman aikaisesti. Näin mutaatiot eivät aiheuta niin sanottuja kilpailutilanteita (engl. race condition) ja mahdollista tästä johtuvaa datan tai palvelimen tilan korruptoitumista. \cite{QueriesAndMutations, graphql-spec}

%--------------- KAPPALE ---------------------%

\section{REST}

Representational State Transfer (REST) on vuonna 2000 julkaistussa Roy Fieldingin väitöskirjassa määritelty rajapinta-arkkitehtuurimalli hajautettuja hypermediasysteemejä varten. Motivaattorina REST:n kehittämiselle oli tarve luoda arkkitehtuurinen malli WWW:n toiminnalle. REST määrittelee erilaisten rajoitteiden kautta, kuinka skaalautuvia ja pitkällä aikavälillä helposti ylläpidettäviä rajapintoja tulisi suunnitella. \cite{FieldingRThesisREST, FieldingRThesis}

Se mitä REST on ja mitä se ei ole, aiheuttaa hyvin usein hämmennystä rajapintojen parissa työskentelevien ihmisten keskuudessa. Tämä johtunee siitä, että REST ei sinänsä ole täydellinen arkkitehtuurimalli itsessään, vaan joukko kriteereitä näiden suunnittelemiseksi. Kriteerit täyttävät arkkitehtuurimallit taas voivat kutsua itseään REST:ksi. Valitettavasti termiä REST käytetään usein myös rajapinnoista, jotka eivät täytä REST:n alkuperäisiä rajoitteita. Fielding onkin kutsunut REST:iä muotisanaksi (buzzword)\footnote{Yes, REST is a buzzword:  https://mobile.twitter.com/fielding/status/1458499370672791565}, jota käytetään kaikkien hiemankin REST:iä muistuttavien rajapintojen kuvaamiseen. \cite{rest-apis-must-be-hypertext-driven}

REST:n määrittelyä ympäröivän tulkinnanvaraisuuden vuoksi tämän aliluvun tarkoituksena on REST-arkkitehtuurin avaamisen lisäksi määritellä, mitä REST tarkoittaa tässä kandidaatintutkielmassa. Aluksi käydään läpi Roy Fieldingin väitöskirjasta löytyvät rajoitteet REST-rajapinnoille ja selvitetään missä rajoitteissa REST-rajapinnat epäonnistuvat yleisimmin. Sen jälkeen tehdään katsaus \textit{Richardsonin kypsyysmalliin (Richardson Maturity Model)}, jossa määritellään REST-rajapinnoille kypsyystasot, sen mukaan, mitä rajoitteita nämä toteuttavat. Seuraavaksi luodaan REST:lle tämän tutkielman kattava määrittely. Viimeisenä käydään läpi lyhyt käytännön esimerkki REST-rajapintatoteutuksesta.

\subsection{REST-rajapinnan kuusi rajoitetta}
\label{REST-rajapinnan kuusi rajoitetta}

Fieldingiläisen REST-rajapinnan täytyy toteuttaa joukko tarkasti määriteltyjä rajoitteita (engl. constraints). Näin ollen monet REST:ksi itseään kutsuvat rajapinnat eivät aina välttämättä ole REST-rajapintoja. REST:n kaltaisia, vain osan sen rajoitteista toteuttavia rajapintoja, voidaan kutsua REST:n kaltaisiksi rajapinnoiksi (engl. REST-like APIs). \cite{FieldingRThesisREST, rest-apis-must-be-hypertext-driven}

Fieldingin esittämiä REST-rajapintojen rajoitteita on yhteensä kuusi kappaletta, joista yksi on valinnainen, eli sitä ei tarvitse toteuttaa. Käydään seuraavaksi nämä läpi: \cite{FieldingRThesisREST}

\paragraph{Asiakas-palvelin (Client-server)} Ensimmäinen rajoite, jota jokaisen REST-rajapintoja hyödyntävän sovelluksen on seurattava, on asiakas-palvelin-malli. Lyhyesti palvelin tarjoaa resursseja niitä tarvitseville asiakasohjelmille.

\paragraph{Tilattomuus (Stateless)} Asiakas- ja palvelinsovellusten välisen kommunikaation on oltava tilatonta. Jokaisen pyynnön sisältä on löydyttävä kaikki tieto sen ymmärtämiseksi. Tallennetun tilakontekstin hyödyntäminen palvelimella ei ole sallittua: istunnon tila on kokonaisuudessaan asiakasohjelman hallussa. Näin ollen esimerkiksi evästeet (engl. cookies), jotka pitävät kirjaa kirjautumistiedoista palvelimella, rikkovat tätä REST-rajoitetta.

\paragraph{Välimuisti (Cache)} Verkkotehokkuuden parantamiseksi määritellään välimuistirajoite. Hyvin toteutetulla välimuistilla voidaan vähentää palvelimelle tehtyjen pyyntöjen määrää. Jokaisen pyynnön vastauksen yhteydessä, data voidaan määritellä tallennettavaksi välimuistiin. Tällöin asiakasohjelma voi käyttää tätä dataa tulevien vastaavien pyyntöjen täyttämiseen.

\paragraph{Yhtenäinen rajapinta (Uniform Interface)} Yhtenäisen rajapinnan rajoitteen kautta pyritään yksinkertaistamaan järjestelmän arkkitehtuuria, tarjoamaan näkyvyyttä vuorovaikutustilanteissa sekä eristämään rajapinnat asiakasohjelmista, jotta kummatkin voivat kehittyä erillään toisistaan. Fielding määritteli yhtenäisen rajapinnan toteutumiseksi vielä neljä rajapintarajoitetta, jotka käydään läpi aliluvussa \ref{REST epäonnistuu usein rajapintarajoitteissa}.

\paragraph{Kerroksittainen järjestelmä (Layered System)} Kerroksittaisessa järjestelmässä sovelluksen palvelut on järjestetty hierarkisesti: jokainen kerros tarjoaa palveluita ylemmälle kerrokselle ja ottaa vastaan palveluita alemmalta kerrokselta \footnote{https://www.ics.uci.edu/~fielding/pubs/dissertation/net\_arch\_styles.htm\#sec\_3\_4\_2}.

\paragraph{Ladattava koodi (Code-On-Demand)} Viimeinen ja ainoa valinnainen rajoite REST-rajapinnoille. Asiakasohjelmien toiminnallisuutta voidaan laajentaa lataamalla ja suorittamalla palvelimelta haettua koodia, esimerkiksi applettien tai skriptien muodossa. Tämä rajoite on valinnainen, sillä koodin lataamista ei voida välttämättä toteuttaa kaikissa tilanteissa näkyvyyden johdosta. Organisaation palomuuri voi esimerkiksi estää Java-applettien lataamisen ulkoisista lähteistä, jolloin nämä rajapinnat näennäisesti eivät toteuttaisi ladattavan koodin rajoitetta.

\subsection{REST epäonnistuu usein rajapintarajoitteissa}
\label{REST epäonnistuu usein rajapintarajoitteissa}

Yhteinäisen rajapinnan rajoite toteutuu, kun se täyttää neljä seuraavaa ominaisuutta: resursseilla on oltava tunnisteet, resursseja manipuloidaan representaatioiden kautta, viestien täytyy olla "itsensä kuvaavia" (self-descriptive) ja hypermediaa on käytettävä sovellustilan moottorina (HATEOAS). \cite{FieldingRThesisREST}

Termejä avataan enemmän Fieldingin väitöskirjan aliluvussa 5.2 \footnote{https://www.ics.uci.edu/~fielding/pubs/dissertation/rest\_arch\_style.htm\#sec\_5\_2}. REST:n kaltaisissa rajapinnoissa näistä rajapintarajoitteista jää usein toteutumatta HATEOAS-rajoite. HATEOAS-rajoitteen mukaan jokaisen rajapinnan vastauksen yhteydessä on palautettava lista linkkejä, joista selviää minkälaisia jatkotoimenpiteitä rajapintaan voidaan tehdä. Näin ollen rajapintaa käyttävien asiakasohjelmien tulisi pystyä käyttämään rajapintaa jo pelkästään tietämällä sen juuriosoite. Rajapinta käyttäytyy tällöin verkkosivun lailla, jota voidaan navigoida linkkien avulla. \cite{rest-apis-must-be-hypertext-driven}

HATEOAS-rajapintarajoite mahdollistaa asiakasohjelmien ja rajapinnan itsenäisen kehittymisen. Rajapintaa ei myöskään vältämättä tarvitse versioida, sillä asiakasohjelmaan ei (toivottavasti) olla kovakoodattu muita osoitteita kuin juuriosoite \footnote{https://alexkondov.com/tao-of-node/\#use-hypermedia}. 

\subsection{REST tämän tutkielman yhteydessä}
\label{REST tämän tutkielman yhteydessä}

Vaikka REST-termille on vuosien varrella kehittynyt hyvinkin joustava ja tilanteesta riippuva määrittely, pyritään tässä tutkielmassa pääasiassa noudattamaan alkuperäistä fieldingiläistä määritelmää REST-rajapinnalle. Tämä tutkielma huomioi kuitenkin vertailuissa myös \textit{Richardsonin kypsyysasteikon} tason 2 REST-rajapinnat, sillä käytännössä suurin osa maailman "REST-rajapinnoista" luultavasti yltää vain tälle tasolle \cite{rmm, tao-of-node, fullstack-rest}. Jos tutkielma poikkeaa fieldingiläisestä määritelmästä, mainitaan tästä selkeästi.

Richardsonin malliin voi tutustua tarkemmin esimerkiksi   \href{https://martinfowler.com/articles/richardsonMaturityModel.html}{täällä} \footnote{https://martinfowler.com/articles/richardsonMaturityModel.html}. Se kuvaa REST-rajapinnan kypsyyttä, sen toteuttamien rajoitteiden määrän kautta. Taso 2 mallissa tarkoittaa lähinnä sellaista REST-rajapintaa, josta puuttuu yhtenäisen rajapinnan HATEOAS-rajoite. \cite{rmm}

\begin{table}[h!]
\begin{tabular}{|l|l|l|}
\hline
\textbf{URL}  & \textbf{verbi} & \textbf{toiminnallisuus}                                          \\ \hline
accounts     & GET            & hakee kaikki kokoelman resurssit                                 \\ \hline
accounts/100 & GET            & hakee yksittäisen kokoelman resurssin                            \\ \hline
accounts     & POST           & luo uuden resurssin pyynnön mukana olevasta datasta              \\ \hline
accounts/100 & DELETE          & poistaa yksilöidyn resurssin                                     \\ \hline
accounts/100 & PUT            & korvaa yksilöidyn resurssin pyynnön mukana olevalla datalla      \\ \hline
accounts/100 & PATCH          & korvaa yksilöidyn resurssin osan pyynnön mukana olevalla datalla \\ \hline
\end{tabular}
\caption{Resursseille voidaan suorittaa erityyppisiä operaatioita. Operaation määrittelee HTTP-operaation tyyppi tai verbi.}
\label{table:1}
\end{table}

Käytännön toteutuksia vertailtaessa, tässä tutkielmassa seurataan Leonard Richardsonin ja Sam Rubyn \cite{restful-web-services} kirjassa \textit{RESTful Web Services} määriteltyä yleistä tapaa toteuttaa REST-rajapintoja. He ovat nimenneet arkkitehtuurityylinsä \textit{Resurssipohjaiseksi arkkitehtuurityyliksi} (engl. Resource-Oriented Architecture). Yksinkertainen toteutus voidaan nähdä taulukosta \ref{table:1}\footnote{https://fullstackopen.com/osa3/node\_js\_ja\_express\#rest}. Resursseja yksilöivä URL muodostetaan resurssityypin nimen ja sen yksilöivän tunnisteen mukaan. Jos palvelun juuriosoite on \textit{www.bank.com}, niin GET-pyyntö osoitteeseen \textit{www.bank.com/accounts/12345} palauttaa resurssin, jonka id on 12345. Vastaus voisi näyttää seuraavalta: 

\inputminted{json}{listaukset/rest.json}

Resurssin tietojen lisäksi vastaus sisältää myös listan seuraavista mahdollisista toiminnoista resurssille. Seuraavat toiminnot on listattava, jotta HATEOAS-rajoite täyttyy.

\inputminted{json}{listaukset/rest2.json}

Mikäli sovelluksen tila vaihtuu, myös vastauksen mukana tulevien linkkien määrä vaihtelee. Jos tämän satunnaisen pankin tili näyttää negatiivista, vastauksen mukana tulisi luultavasti vain linkki talletukseen. Nostoa ja tilisiirtoja ei voitaisi tehdä. \cite{restful-web-services, rest-apis-must-be-hypertext-driven}



\chapter{GraphQL ja REST: miten ne vertautuvat toisiinsa?} \label{kolmasluku}

Tässä luvussä vertaillaan GraphQL- ja REST-rajapintoja. Vertailun kohteina ovat esimerkiksi käytännön toteutus, arkkitehtuurien tehokkuus erilaisissa käyttötilanteissa ja turvallisuus. Tämä luku pyrkii vastaamaan ensimmäiseen tutkielman tutkimuskysymykseen \textit{mitä hyötyjä ja haittoja on GraphQL:n käytössä verrattuna REST:iin?}

\setlength{\parindent}{0em}
\setlength{\parskip}{1em}

\section{Käytännön toteutus}

Käytännön toteutus on yksi suurimmista GraphQL- ja REST-rajapintojen välisistä eroista. GraphQL määrittelee hyvin selkeästi toteutuksen suuntaviivat kyselykielelleen ja moottorilleen. Näitä on seurattava tarkasti, jotta GraphQL-toteutusta voi kutsua GraphQL:ksi. REST sen sijaan määrittelee suuntaa-antavat rajoitteet, mutta jättää myös hyvin paljon asioita ohjelmoijan tulkittavaksi: resurssien osoitteet, datan muoto (XML, JSON jne.), käyttöympäristö. REST-palvelun luominen ei siis ole välttämättä niin suoraviivaista ja mahdollisia toteutuksia on paljon. 

Kun rajapintaa lähdetään rakentamaan, GraphQL:lle löytyy useita erilaisia valmiita toteutuksia kirjastojen muodossa. Ohjelmoijan ei siis usein tarvitse itse lähteä toteuttamaan moottoria ja kyselykieltä. Esimerkiksi JavaScriptille löytyy hyvin suosittu \href{https://www.apollographql.com/}{Apollo}-kirjasto \footnote{https://www.apollographql.com/} sekä palvelimen että asiakas-sovelluksen GraphQL tarpeisiin. Apollon mukana tulee myös monia lisäominaisuuksia, joita ei olla määritelty GraphQL-spesifikaatiossa. Näitä ovat esimerkiksi asiakkaan ja palvelimen välimuistin hallinta ja joukko uusia direktiivejä. Lista suosituista GraphQL-toteutuksista useille kielille löytyy \href{https://graphql.org/code/}{GraphQL}:n sivuilta \footnote{https://graphql.org/code/}. \cite{apollo-docs, graphqlorg}

Täydelliselle RESTful-toteutukselle on vaikeampi löytää yhtä yksittäistä kirjastoa, sillä toteutus riippuu hyvin paljon ohjelmoijan omista valinnoista. Mikäli etsitään jälleen toteutuksia JavaScript-kielelle Node.js-ympäristössä ja data halutaan tarjota JSON-muodossa, REST rajapinta toteutetaan mitä luultavimmin suositun \href{https://expressjs.com/}{Express}-kirjaston \footnote{https://expressjs.com/} päälle. Express ei kuitenkaan tarjoa selkeää toteutusta kaikille REST-rajapintojen rajoitteille, kuten HATEOAS. Rajoite on joko toteutettava itse, tai sen toteuttamiseen on etsittävä oma kirjastonsa. \cite{express}

%toteutuksen helppous uudestaan

\section{Versiointi ja dokumentaatio}

GraphQL ja REST-rajapintoja ei tarvitse versioida. GraphQL:n ideana on versioinnin sijaan ylläpitää yhtä kehittyvää versiota rajapinnasta. Vanhentuneet kentät voidaan merkitä poistettavaksi käytöstä ja piilottaa työkaluista \cite{graphqlorg}. REST ei sen sijaan ota selvää kantaa versiointiin, mutta mikäli HATEOAS-rajoite toteutuu, versiointi on hyvin varmasti turhaa. Kun puhutaan GraphQL:n versioinnin puuttumisen eduista verrattuna REST:iin, viitataan hyvin luultavasti REST:n kaltaisiin rajapintoihin, joissa ei toteudu HATEOAS-rajoite.

GraphQL-projekteissa käytetään dokumentaatioon useimmiten \textbf{GraphiQL}-työkalua, joka on GraphQL-säätiön alaisuudessa oleva virallinen projekti. GraphiQL paljastaa rajapinnan datan selaimessa olevan käyttöliittymän kautta ja se tulee mukana useimmissa GraphQL-toteutuksissa. GraphQL-rajapinnalle ei siis yleensä tarvitse itse kirjoittaaa kaiken kattavaa dokumentaatiota. 

REST ei määrittele yksittäistä tapaa rajapinnan dokumentaatioon. Monimutkaista dokumentaatiota ei yleensä tarvita myöskään REST:iä käytettäessä, sillä RESTful-rajapinta on luonteeltaan itsensä dokumentoiva HATEOAS-rajoitteen ansiosta. Mikäli HATEOAS-rajoite on kuitenkin jätetty pois, jonkinlainen dokumentaatio on hyvä olla olemassa ja se on luultavasti luotava käsin.

\section{Turvallisuus}

GraphQL- ja REST-rajapinnoilla on monia samoja turvallisuusuhkia, jotka ohjelmoijan tulee ottaa huomioon. Injektio- ja palvelunestohyökkäykset sekä rikkinäisen tunnistautumisen käyttö ovat hyvin yleisiä kummankin tyyppisessä rajapinnassa. \cite{owasp-graphql, owasp-rest}

GraphQL-rajapintoja rakennettaessa on myös otettava huomioon GraphQL-kyselykielen tuomat turvallisuusuhat, joita ei esiinny REST-rajapinnoissa. Yleisiä kyselykielen tuomia uhkia ovat "erittelyhyökkäys" (engl. batching attack), syvät kyselyt ja muilla tavoin kuormittavien kyselyiden suorittaminen \cite{owasp-graphql}. Erittelyhyökkäyksessä hyökkääjä tekee useita eri kalliita kyselyitä yhden palvelinpyynnön mukana ja syvät kyselyt käyttävät hyväksi kyselyiden olioita, joiden syvyttä ei olla rajoitettu. GraphQL:n joustavuus määritellä datan muoto antaa myös suuremman hyökkäyskaistan pahaa tarkoittaville tahoille.

GraphQL-kyselykielen tuomat uhat eivät ole ollenkaan triviaaleja. Esimerkiksi vuonna 2019 teetetyssä aiheeseen liittyvässä tutkimuksessa \cite{Wittern2019} huomattiin, että suurin osa GitHub:sta haetuista GraphQL-rajapinnoista sisälsi edellä mainittuja turvallisuusuhkia. Ongelmiin saattaa kuitenkin löytyä ratkaisu. Hartig ja Perez \cite{graphql-analysis} ovat luoneet muodollisen määritelmän GraphQL-kyselykielelle. Tämä pohjalta he huomasivat, että GraphQL-kyselyille voidaan luoda tehokkaita algoritmeja monimutkaisuuden analysointiin \cite{Hartig2018}. Analysoinnilla kehittäjät voivat välttää kyselykielen mukana tulevia turvallisuuspuutteita.

Monessa GraphQL-toteutuksessa on myös oletusarvoisesti joitakin turvattomia konfiguraatiota, joiden tulisi olla pois päältä tuotantoympäristössä. Esimerkiksi GraphiQL ja ylimääräistä tietoa sisältävät virheilmoitukset helpottavat hyökkäyksen kohdistamista GraphQL-rajapintaan. \cite{owasp-graphql}

Näin ollen turvallisten REST-rajapintojen kehittäminen saattaa siis olla helpompaa. Huolellinen ohjelmoija voi kuitenkin minimoida tai poistaa kokonaan GraphQL-kyselykielen ja konfiguraatioiden mukana tulevat turvallisuusuhat. Sekä \href{https://cheatsheetseries.owasp.org/cheatsheets/GraphQL_Cheat_Sheet.html#graphql-cheat-sheet}{GraphQL}\footnote{https://cheatsheetseries.owasp.org/cheatsheets/GraphQL\_Cheat\_Sheet.html\#graphql-cheat-sheet}- että \href{https://cheatsheetseries.owasp.org/cheatsheets/REST_Security_Cheat_Sheet.html#rest-security-cheat-sheet}{REST}\footnote{https://cheatsheetseries.owasp.org/cheatsheets/REST\_Security\_Cheat\_Sheet.html\#rest-security-cheat-sheet}-rajapinnoille löytyy kattavat turvallisuusohjeistukset \textit{OWASP Cheat Sheet} -sarjan verkkosivuilta. \cite{owasp-graphql, owasp-rest}

\section{Tehokkuus eri käyttötilanteissa}

\subsection{Yli- ja alihakeminen}

REST-rajapinnassa jokaisella resurssilla on oltava oma tunnisteensa. Web-ympäristössä tunniste on useimmiten URL ja uniikki kyseiselle resurssille. Tämä tarkoittaa sitä, että REST-rajapinnalla on hyvin varmasti useita eri "päätepisteitä" (engl. endpoints), joista dataa haetaan. GraphQL sen sijaan tarjoaa datansa yhdestä päätepisteestä. Resursseilla ei tarvitse olla URL-tunnisteita, sillä ne haetaan tyyppisysteemin avulla.

Koska REST-määrittelee useimmiten palautettavan datan muodon jäykästi ja pitää tätä tiettyjen URL-tunnisteiden takana, voi monimutkaisissa rajapinnoissa tulla ongelmaksi yli- tai alihakeminen (engl. over- and underfetching): asiakas saa pyynnön mukana joko liian paljon tai vähän dataa. Mikäli palvelin lähettää pyyntöjen mukana ylimääräistä dataa, vastausajoista tulee varmasti pidempiä. Mikäli palvelin ei tarjoa kaikkea dataa yhden pyynnön mukana, asiakkaan on tehtävä useita pyyntöjä palvelimelle, mikä voi johtaa ruuhkautumiseen.

GraphQL ratkaisee REST-rajapinnoissa hyvin yleiset yli- ja alihakemisen ongelmat. Koska asiakas voi määritellä palautettavan datan rakenteen lähes täydellisesti, GraphQL välttää useiden pyyntöjen tekemisen palvelimelle, sekä ylimääräisen datan palauttamisen.

\subsection{Tutkimus: miten helposti REST ja GraphQL rajapintoihin voidaan muodostaa kyselyitä?}

Gleison Brito ja Marco Tulio Valente Brasilian Minas Geraisin liittovaltion yliopistosta esittivät vuonna 2020 tutkimuksen liittyen kyselyiden muodostamiseen GraphQL- ja REST-rajapintoihin. Tutkimuksessa pyrittiin vastaamaan kahteen kysymykseen: \textit{kuinka paljon aikaa kehittäjät käyttävät REST- ja GraphQL-kyselyjen toteuttamiseen} ja \textit{mitkä ovat osallistujien käsitykset REST:stä ja GraphQL:stä}. \cite{Brito2020}

Tutkimukseen osallistui 22 opiskelijaa, joista 10 oli perustutkinto-opiskelijoita ja 12 jatko-opiskelijoita. Kaikilla oli vähintään vuoden verran ohjelmointikokemusta. Tutkimukseen osallistuneista 15 opiskelijalla oli myös kokemusta GraphQL- ja/tai REST-rajapinnoista. \cite{Brito2020}

Tutkimuksen aikana jokainen osallistuja toteutti kyselyitä GitHubin palveluihin sekä REST- että GraphQL-rajapintoja käyttäen. Tulos oli hieman yllättävä: jopa entuudestaan REST-rajapintaa tuntevat ja GraphQL:ää osaamattomat osallistujat pystyivät toteuttamaan kyselynsä selvästi helpommin käyttäen GraphQL:ää. Etenkin kyselyt, joissa oli monta parametriä, olivat selvästi vaikeampia toteuttaa REST-rajapintoihin. Tutkimuksen jälkeen osallistujille tehdyssä kyselyssä mainittiin kaksi selkeää etua GraphQL:n käytölle: työkalutuki kyselyjen rakentamiseen ja testaamiseen (erityisesti automaattisten täydennysominaisuuksien tarjoama tuki) ja tavanomaisia ohjelmointikieliä lähellä oleva syntaksi (rakenteet, kuten oliot, skeemat, tyypit jne.). \cite{Brito2020}

Tutkimuksen perusteella GraphQL vaikuttaisi olevan käyttäjäystävällisempi rajapintoihin kyselyitä toteuttavien ohjelmoijien tarpeisiin. On kuitenkin mahdollista, että tulokset vaihtelisivat, mikäli kyselyitä suoritettaisiin erilaisiin rajapintoihin tai tutkimusta tehtäisiin suuremmalla otannalla. Olisi siis kiinnostavaa nähdä tulevaisuudessa lisää samanlaisia tutkimuksia tulosten vahvistamiseksi.

\subsection{Lisää tutkimuksia REST ja GraphQL rajapintojen tehokkuudesta}

Koska GraphQL on suhteellisen uusi tulokas web-rajapintojen joukossa, REST- ja GraphQL-rajapintojen tehokkuutta vertailevia tutkimuksia ei ole kovinkaan paljoa. Tämän takia tutkimusten tulokset eivät myös ole vielä täysin varmistettuja. Mikäli tutkimusta olisi enemmän, tuloksista voitaisiin olla varmempia. REST ja GraphQL-rajapintojen tehokkuutta saattaa myös olla vaikea vertailla tarkasti, sillä REST antaa itselleen vain suunta-antavat rajoitteet. Näin ollen kaksi REST-toteutusta voivat olla hyvin erilaisia.

Tehokkuutta vertailevissa tutkimuksissa on joitakin yleisiä johtopäätöksiä. GraphQL vaikuttaa olevan tehokkaampi, kun suoritetaan monimutkaisia kyselyitä ja REST-pohjainen rajapinta näyttää palauttavan datan nopeammin mikäli dataa on paljon \cite{mateusz2020, Seabra2019, helgason2017}. Toisaalta Vadlamani \cite{vadlamani2021} toteaa tutkimuksessaan, että REST ja GraphQL-rajapintojen välisessä tehokkuudessa ei ole tilastollisesti kovinkaan paljoa eroa mikäli rajapintaan tehdään yksittäisiä kyselyitä. Joka tapauksessa tulosten vahvistamiseksi olisi kuitenkin hyvä tehdä lisää tutkimustyötä.

\section{Suosio}

REST ja GraphQL ovat molemmat suosittuja ratkaisuja rajapintojen rakentamiseen. REST- ja REST:n kaltaiset rajapinnat ovat kuitenkin selvästi yleisempiä kuin GraphQL-rajapinnat. Monessa rajapintojen parissa työskenteleville kehittäjille teetetyssä kyselyssä, REST-rajapintojen käyttöaste on noin 90\% tienoilla, kun taas GraphQL:ää sanoo käyttävänsä noin 30-40\% kehittäjistä. Koska REST-ekosysteemi on iäkkäämpi ja ainakin vielä toistaiseksi suositumpi, löytyy sille hyvin luultavasti enemmän työkaluja ja yhteisön tuottamaa tietoa ongelmien ratkaisemiseksi.

GraphQL:n suosion kasvu näytti olevan tasaista vuoteen 2019 asti. Sen jälkeen GraphQL:n käytön kasvu on kuitenkin pysähtynyt, eikä suosio ole näyttänyt nousseen juurikaan vuosina 2020 ja 2021. Tästä huolimatta Vadlamanin \cite{vadlamani2021} teettämän tutkimuksen mukaan yli puolet kehittäjistä uskoo GraphQL:n saavan lisää suosiota seuraavan viiden vuoden aikana. \cite{statejs-report, smartbear-report, postman-report}

Kyselyiden pohjalta on kuitenkin vaikea sanoa, ovatko käytetyt REST-rajapinnat oikeasti RESTful. On hyvinkin mahdollista, että GraphQL:n suosio on suurempaa kuin RESTful rajapintojen. Väitettä on kuitenkin vaikea todentaa, sillä hyvin moni REST-rajapinta ei ole julkisesti nähtävissä, eikä niide fieldingiläisyyttä voida näin tarkistaa. Voidaan kuitenkin todeta, että ainakin suurimmaksi osaksi REST-periaatteita noudattavat rajapinnat eivät ole häviämässä vielä minnekkään.

\section{Yhteenveto}

Tämän luvun tulosten perusteella sekä GraphQL- että REST-rajapinnoilla on selviä heikkouksia ja vahvuuksia. GraphQL suoriutuu selvästi paremmin, kun suoritetaan monimutkaisia kyselyitä. Kyselykielen avulla voidaan välttää datan yli- ja alihakeminen. GraphQL-kyselykielen tuoma joustavuus antaa asiakkaalle mahdollisuuden hakea dataa juuri sopivassa muodossa. Näin ollen GraphQL-rajapinnan on myös helpompi tajota dataa erilaisille asiakasohjelmille. GraphQL saattaa olla myös tehokkaampi hitaammissa verkkoyhteyksissä, sillä useita pyyntöjä palvelimelle ei tarvita, jotta haluttu vastaus voidaan muodostaa.

REST vaikuttaa olevan GraphQL:ää turvallisempi ratkaisu, koska GraphQL-kyselykielen tuoma joustavuus antaa hyökkääjälle myös enemmän mahdollisuuksia hyväksikäyttää GraphQL-rajapintoja. REST on myös paljon suositumpi ja iäkkäämpi rajapintamalli. Sen ekosysteemi, työkalut ja kirjastot ovat varttuneempia ja sille löytyy enemmän osaavia kehittäjiä.\chapter{Johtopäätökset} \label{neljasluku}

\setlength{\parindent}{0em}
\setlength{\parskip}{1em}

Ohjelmointirajapinnat ovat olleet keskeisessä asemassa web-sovelluksissa ja niiden kehityksessä kautta internetin historian. Kaikki yksinkertaisia staattisia sivuja monimutkaisemmin dataa tarjoavien verkkosivujen taustalla on lähes poikkeuksetta jonkinlainen rajapinta, josta data haetaan. Kehittäjiä kohdanneet suurimmat ongelmat rajapintoja kehittäessä ovat olleet asiakas- ja palvelinsovellusten eristäminen selkeästi omiksi yksiköikseen, jotta nämä voivat kehittyä itsenäisesti erillään toisistaan, sekä datanhaun tehokas kaistanleveyden käyttö, jotta sovellus toimii myös hitaammissa verkkoyhteyksissä.

GraphQL on noussut suosituksi vaihtoehdoksi perinteisemmälle REST-rajapinnalle viimeisen vuosikymmenen aikana. Tietyissä tilanteissa GraphQL tarjoaa parempia ratkaisuja kehittäjiä vaivaaviin yleisiin ongelmiin, jotka liittyvät datan hakemiseen rajapinnasta tehokkaasti, nopeasti ja intuitiivisesti.

GraphQL ei kuitenkaan näytä vielä ainakaan lähivuosina syrjäyttävän REST-ekosysteemiä eikä GraphQL:n suosion kasvu ole ollut kenties niin nopeaa kuin on oletettu. Rajapintoja vertailevassa keskustelussa on myös unohdettu, että REST on muodostunut muotisanaksi, jolla viitataan usein myös rajapintoihin, jotka eivät ole "RESTful" termin alkuperäisessä merkityksessä. Saattaakin olla, että GraphQL on nykyään suositumpi kuin REST-rajapinnat. Väitettä on kuitenkin vaikea todistaa, sillä hyvin iso osa rajapinnoista ei ole julkisesti näkyvillä.

On kuitenkin melko varmaa, että REST:n kaltaiset rajapinnat eivät ole toistaiseksi häviämässä lähivuosina mihinkään. Vastaus tutkimuskysymykseen \textit{tuleeko GraphQL korvaamaan REST-rajapinnat} on siis seuraava: GraphQL saattaa olla jo suositumpi kuin RESTful rajapinnat, mutta GraphQL ei tule korvaamaan REST:n kaltaisia rajapintoja ainakaan vielä lähivuosina.

Lisää REST- ja GraphQL-rajapintoja vertailevaa tutkimustyötä voitaisiin tehdä vielä tulevaisuudessakin. Tutkimusta on tehty vasta hyvin vähän, eikä tuloksiin näin ollen voida luottaa täysin varmasti. Etenkin GraphQL:n ja REST:in turvallisuutta ja käyttökokemusta vertailevien tutkimusten tulosten vahvistuminen tulevalla työllä olisi kiinnostavaa.

%\input{file_name_of_chapter_x}
%\input{file_name_of_chapter_y}

\begin{comment}
The thesis main content ends here.
\end{comment}
\printbibliography

\begin{comment}
Create your appendix chapters with command \textbackslash appchapter\{some
name\} instead of \textbackslash chapter\{some name\} for the automagic
page counting to work!
\end{comment}


\begin{comment}
main document ends here
\end{comment}

\end{document}
