\chapter{Johtopäätökset} \label{neljasluku}

\setlength{\parindent}{0em}
\setlength{\parskip}{1em}

Ohjelmointirajapinnat ovat olleet keskeisessä asemassa web-sovelluksissa ja niiden kehityksessä kautta internetin historian. Kaikki yksinkertaisia staattisia sivuja monimutkaisemmin dataa tarjoavien verkkosivujen taustalla on lähes poikkeuksetta jonkinlainen rajapinta, josta data haetaan. Kehittäjiä kohdanneet suurimmat ongelmat rajapintoja kehittäessä ovat olleet asiakas- ja palvelinsovellusten eristäminen selkeästi omiksi yksiköikseen, jotta nämä voivat kehittyä itsenäisesti erillään toisistaan, sekä datanhaun tehokas kaistanleveyden käyttö, jotta sovellus toimii myös hitaammissa verkkoyhteyksissä.

GraphQL on noussut suosituksi vaihtoehdoksi perinteisemmälle REST-rajapinnalle viimeisen vuosikymmenen aikana. Tietyissä tilanteissa GraphQL tarjoaa parempia ratkaisuja kehittäjiä vaivaaviin yleisiin ongelmiin, jotka liittyvät datan hakemiseen rajapinnasta tehokkaasti, nopeasti ja intuitiivisesti.

GraphQL ei kuitenkaan näytä vielä ainakaan lähivuosina syrjäyttävän REST-ekosysteemiä eikä GraphQL:n suosion kasvu ole ollut kenties niin nopeaa kuin on oletettu. Rajapintoja vertailevassa keskustelussa on myös unohdettu, että REST on muodostunut muotisanaksi, jolla viitataan usein myös rajapintoihin, jotka eivät ole "RESTful" termin alkuperäisessä merkityksessä. Saattaakin olla, että GraphQL on nykyään suositumpi kuin REST-rajapinnat. Väitettä on kuitenkin vaikea todistaa, sillä hyvin iso osa rajapinnoista ei ole julkisesti näkyvillä.

On kuitenkin melko varmaa, että REST:n kaltaiset rajapinnat eivät ole toistaiseksi häviämässä lähivuosina mihinkään. Vastaus tutkimuskysymykseen \textit{tuleeko GraphQL korvaamaan REST-rajapinnat} on siis seuraava: GraphQL saattaa olla jo suositumpi kuin RESTful rajapinnat, mutta GraphQL ei tule korvaamaan REST:n kaltaisia rajapintoja ainakaan vielä lähivuosina.

Lisää REST- ja GraphQL-rajapintoja vertailevaa tutkimustyötä voitaisiin tehdä vielä tulevaisuudessakin. Tutkimusta on tehty vasta hyvin vähän, eikä tuloksiin näin ollen voida luottaa täysin varmasti. Etenkin GraphQL:n ja REST:in turvallisuutta ja käyttökokemusta vertailevien tutkimusten tuulosten vahvistuminen tulevalla työllä olisi kiinnostavaa.