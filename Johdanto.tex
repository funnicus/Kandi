\chapter{Johdanto} \label{Johdanto}

\setlength{\parindent}{0em}
\setlength{\parskip}{1em}

Roy Fielding esitti vuonna 2000 väitöskirjassaan \textit{Architectural Styles and the Design of Network-based Software Architectures} \textbf{REST} (Representational State Transfer) rajapinta-arkkitehtuurin. REST määritteli, miten dataa (REST-maailmassa resursseja), kuten HTML, JSON ja XML, tulisi jakaa WWW:n kaltaisessa suuressa hypermediasysteemissä \cite{FieldingRThesis}. Läpi 90-luvun oli suuri tarve määritellä yhteinen standardi verkon rajapinnoille, jonka ratkaisuksi Fielding yritti tarjota REST:iä \cite{FactsAboutW3C}. Sittemmin REST on noussut hyvin suosituksi tavaksi luoda rajapintoja web-ympäristössä mm. helpon toteuttettavuudensa takia.

Facebookin (nykyisin Meta) sisäisesti vuonna 2012 kehittämä GraphQL on noussut viime vuosina REST:n valta-aseman todennäköisimmäksi haastajaksi \cite{graphql-intro}. Jopa REST:n \href{https://www.stridenyc.com/podcasts/52-is-2018-the-year-graphql-kills-rest}{kuolemaa} on ennustettu aika ajoin, mutta GraphQL:n voitto kamppailussa ei näytä vielä olevan aivan täysin taattua \cite{the-year-graphql-kills-rest}. Kysymys kuuluukin: johtaako GraphQL todellakin REST-arkkitehtuurin käytön hiipumiseen vai tulevatko nämä elämään rinnakkain vaihtoehtoisina tapoina toteuttaa rajapintoja verkossa?

%--------------- KAPPALE ---------------------%

\section{Tutkimuksen rajaus}

Tässä kandidaatintutkielmassa tarkastellaan kahta web rajapinta-arkkitehtuurimallia pääasiallisesti HTTP-protokollan yli toteutettavalle kommunikoinnille: \textbf{GraphQL} ja \textbf{REST}. Muitakin samanlaisia arkkitehtuurimalleja kuten \textbf{SOAP} tai \textbf{RPC} on olemassa, mutta näitä malleja ei käydä läpi tässä tutkielmassa. Tämä työ keskittyy puhtaasti REST:n ja GraphQL:n vertailuun, sillä ne ovat tällä hetkellä ajankohtaisimmat ja suosituimmat arkkitehtuurimallit, verkon asiakas-palvelinsovelluksia rakennettaessa.

GraphQL ja REST ovat jo itsessään niin laajoja käsitteitä, että niistä voisi tehdä kokonaan omat tutkielmansa. Tässä työssä pyritään käymään läpi REST ja GraphQL vain vertailulle oleellisella tavalla. Esimerkiksi GraphQL-moottorin tarkempi toteutus on jätetty tästä tutkielmasta kokonaan pois. REST:iin ja GraphQL:ään voi syvällisemmin tutustua esimerkiksi Roy Fieldingin \href{https://www.ics.uci.edu/~fielding/pubs/dissertation/top.htm}{väitöskirjassa}\footnote{https://www.ics.uci.edu/~fielding/pubs/dissertation/top.htm} (REST) ja GraphQL:n virallisilla \href{https://graphql.org/}{sivuilla}\footnote{https://graphql.org/} sekä GraphQL:n \href{https://spec.graphql.org/}{spesifikaatioissa}\footnote{https://spec.graphql.org/} (GraphQL).

%--------------- KAPPALE ---------------------%

\section{Tutkimuskysymykset}\label{Alaotsikko}

Työn rajauksen perusteella asetetttiin seuraavat tutkimuskysymykset:

\subsubsection{Mitä hyötyjä ja haittoja on GraphQL:n käytössä verrattuna REST:iin?}

Tämän tutkimuskysymyksen puitteissa pyritään vertailemaan GraphQL:ää ja REST:iä. Vertailu luo pohjan työn lopputulokselle ja seuraavaksi esiteltävän, työn kannalta oleellisimman, tutkimuskysymyksen vastaamiseen.


\subsubsection{Tuleeko GraphQL korvaamaan REST-rajapinnat?}

Tämän kandityön lopputuloksen kannalta tärkein kysymys. Tässä työssä halutaan selvittää, tuleeko GraphQL ylittämään REST:n suosion ratkaisuna verkon rajapinnoille.


%--------------- KAPPALE ---------------------%

\section{Tutkimusmenetelmät ja lähteet}

Kun käsitellään verkkoympäristössä eläviä, nopeasti kehittyviä teknologioita ja arkkitehtuurimalleja, uusimman ja luotettavimman tiedon löytää usein aihetta edustavilta virallisilta verkkosivuilta. Itse REST ja GraphQL-rajapintoja vertailevaa tutkimustyötä ole tuotettu vielä kovinkaan paljoa ja näin ollen työssä käytettävien tieteellisten lähteiden määrä on vähäisempi ja niiden esittämiin väitteisiin on suhtauduttava suuremmalla varauksella kuin vastaavissa tutkielmissa yleensä.

Tämä työ käyttää REST:iä esittävän informaation pääasiallisena lähteenä Roy Fielding väitöskirjaa \textit{Architectural Styles and the Design of Network-based Software Architectures} \cite{FieldingRThesis}. GraphQL:ää edustava tieto on pääasiassa peräisin graphql.org-verkkosivulta \cite{graphqlorg}. Lisäksi näiden tukena käytetään lähteenä web-standardien saralla luotettaviksi todettuja verkkolähteitä kuten MDN Webdocs ja w3.org \cite{mdn, FactsAboutW3C}. Tieteellisiä lähteitä on haettu Google Scholarista ja IEEE:n sekä ACM:n tietokannoista.